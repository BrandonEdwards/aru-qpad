\documentclass[]{article}

\usepackage{natbib}
\bibliographystyle{unsrtnat}

%opening
\title{Estimating landbird detection probabilities from autonomous recording unit data}
\author{
	Edwards, Brandon P.M.\\
	\and
	Knight, Elly C.\\
	\and
	Docherty, Teegan D.S.\\
	\and
	Hedley, Richard W.\\
	\and
	Bennett, Joseph R.\\
	\and
	Smith, Adam C.\\
	\and
	Bayne, Erin\\
}

\begin{document}

\maketitle

\begin{abstract}

The NA-POPS project is an initiative to estimate detection probabilities for all landbird species in North America. These detection probabilities, which are the product of a bird's availability and perceptibility, can be used to integrated disparate datasets into a common standard, generate density and population estimates, or inform value of information analyses. The probabilities are estimated from human-conducted point counts via either removal sampling, distance sampling, or both. Autonomous recording units (ARUs) are increasingly used for acoustic monitoring of wildlife, especially in remote regions such as Canada's boreal forest, where human surveys are more expensive. Hundreds of thousands of ARUs deployed across Canada have collected data on population of hundrds of species of birds by conventional means. Recently, methods have been developed for analyzing ARU data to estimate the direction and distance to a singing bird by making use of local...

\end{abstract}

\section{Introduction}
Detectability is an important metric for estimating density of animals \citep{solymos_lessons_2020}, 

\bibliography{refs}
\end{document}
